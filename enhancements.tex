\documentclass[11pt]{article}
\usepackage{amsmath}
\usepackage{amsfonts}
\usepackage{amsthm}
\usepackage[utf8]{inputenc}
\usepackage[margin=0.75in]{geometry}

\title{CSC111 Winter 2024 Project 1}
\author{Benjamin and Neyl}
\date{\today}

\begin{document}
\maketitle

\section*{Enhancements}


\begin{enumerate}

\item Describe your enhancement \#1 here
	\begin{itemize}
	\item For this enhancement, we added music to our game that changes once you arrive at a certain location.
	\item Complexity level: Medium
	\item Adding music in of itself was of medium complexity, because we had to learn how to use the pygame.mixer add on and
	\item how to toggle one music on and another off. The implementation of this enhancement was
	\item not very complex, including just several lines of code. The major issue that we encountered was when the first track played,
	\item and when the second track toggled, they both played at the same time instead of switching.

	\end{itemize}

\item Describe your enhancement \#2 here
	\begin{itemize}
	\item Our second enhancement was a secret shortcut to win the game without having to collect
	\item any items thanks to a special guest. This shortcut was found by following a particularly challenging path in the map.
	\item Complexity level: High
	\item This shortcut used inheritance by creating a child class of Location, called "pn_tower". This
	\item child had an additional method which changed the music(from enhancement 1) and was used to create
	\item the pn_tower secret location in the method get_location. This method used "isinstance" to determine if
	\item this method should be called, and again to determine if the extra dialogue should be triggered.
	\item This enhancement also required a lot of extra dialogue in the game and was tedious to complete.

	\end{itemize}

\end{enumerate}


\section*{Extra Gameplay Files}

gameplay1.txt
Shows gameplay of the player finding the secret shortcut to the end of the game by visiting the "pn_tower" location.

\end{document}
